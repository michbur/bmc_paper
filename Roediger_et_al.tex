%% BioMed_Central_Tex_Template_v1.06
%%                                      %
%  bmc_article.tex            ver: 1.06 %
%                                       %

%%IMPORTANT: do not delete the first line of this template
%%It must be present to enable the BMC Submission system to
%%recognise this template!!

%%%%%%%%%%%%%%%%%%%%%%%%%%%%%%%%%%%%%%%%%
%%                                     %%
%%  LaTeX template for BioMed Central  %%
%%     journal article submissions     %%
%%                                     %%
%%          <8 June 2012>              %%
%%                                     %%
%%                                     %%
%%%%%%%%%%%%%%%%%%%%%%%%%%%%%%%%%%%%%%%%%


%%%%%%%%%%%%%%%%%%%%%%%%%%%%%%%%%%%%%%%%%%%%%%%%%%%%%%%%%%%%%%%%%%%%%
%%                                                                 %%
%% For instructions on how to fill out this Tex template           %%
%% document please refer to Readme.html and the instructions for   %%
%% authors page on the biomed central website                      %%
%% http://www.biomedcentral.com/info/authors/                      %%
%%                                                                 %%
%% Please do not use \input{...} to include other tex files.       %%
%% Submit your LaTeX manuscript as one .tex document.              %%
%%                                                                 %%
%% All additional figures and files should be attached             %%
%% separately and not embedded in the \TeX\ document itself.       %%
%%                                                                 %%
%% BioMed Central currently use the MikTex distribution of         %%
%% TeX for Windows) of TeX and LaTeX.  This is available from      %%
%% http://www.miktex.org                                           %%
%%                                                                 %%
%%%%%%%%%%%%%%%%%%%%%%%%%%%%%%%%%%%%%%%%%%%%%%%%%%%%%%%%%%%%%%%%%%%%%

%%% additional documentclass options:
%  [doublespacing]
%  [linenumbers]   - put the line numbers on margins

%%% loading packages, author definitions
\usepackage{xr}
\externaldocument{chipPCR}

\documentclass[twocolumn]{bmcart}% uncomment this for twocolumn layout and 
%comment line below
%\documentclass{bmcart}

%%% Load packages
%\usepackage{amsthm,amsmath}
%\RequirePackage{natbib}
\RequirePackage{hyperref}
\usepackage[utf8]{inputenc} %unicode support
\usepackage{listings}
%\usepackage[applemac]{inputenc} %applemac support if unicode package fails
%\usepackage[latin1]{inputenc} %UNIX support if unicode package fails


%%%%%%%%%%%%%%%%%%%%%%%%%%%%%%%%%%%%%%%%%%%%%%%%%
%%                                             %%
%%  If you wish to display your graphics for   %%
%%  your own use using includegraphic or       %%
%%  includegraphics, then comment out the      %%
%%  following two lines of code.               %%
%%  NB: These line *must* be included when     %%
%%  submitting to BMC.                         %%
%%  All figure files must be submitted as      %%
%%  separate graphics through the BMC          %%
%%  submission process, not included in the    %%
%%  submitted article.                         %%
%%                                             %%
%%%%%%%%%%%%%%%%%%%%%%%%%%%%%%%%%%%%%%%%%%%%%%%%%


\def\includegraphic{}
\def\includegraphics{}



%%% Put your definitions there:
\startlocaldefs
\endlocaldefs


%%% Begin ...
\begin{document}

%%% Start of article front matter
\begin{frontmatter}

\begin{fmbox}
\dochead{Software}

%%%%%%%%%%%%%%%%%%%%%%%%%%%%%%%%%%%%%%%%%%%%%%
%%                                          %%
%% Enter the title of your article here     %%
%%                                          %%
%%%%%%%%%%%%%%%%%%%%%%%%%%%%%%%%%%%%%%%%%%%%%%

\title{chipPCR: an R Package to Pre-Process Amplification Curve Data}

%%%%%%%%%%%%%%%%%%%%%%%%%%%%%%%%%%%%%%%%%%%%%%
%%                                          %%
%% Enter the authors here                   %%
%%                                          %%
%% Specify information, if available,       %%
%% in the form:                             %%
%%   <key>={<id1>,<id2>}                    %%
%%   <key>=                                 %%
%% Comment or delete the keys which are     %%
%% not used. Repeat \author command as much %%
%% as required.                             %%
%%                                          %%
%%%%%%%%%%%%%%%%%%%%%%%%%%%%%%%%%%%%%%%%%%%%%%

\author[
   addressref={aff1},                   % id's of addresses, e.g. {aff1,aff2}
   corref={aff1},                       % id of corresponding address, if any
   email={stefan.roediger@hs-lausitz.de}   % email address
]{\inits{S}\fnm{Stefan} \snm{R\"odiger}}
\author[
   addressref={aff2},
   email={michalburdukiewicz@gmail.com}
]{\inits{M}\fnm{Micha\l{}} \snm{Burdukiewicz}}
\author[
   addressref={aff1},
   email={peter.schierack@hs-lausitz.de}
]{\inits{P}\fnm{Peter} \snm{Schierack}}

%%%%%%%%%%%%%%%%%%%%%%%%%%%%%%%%%%%%%%%%%%%%%%
%%                                          %%
%% Enter the authors' addresses here        %%
%%                                          %%
%% Repeat \address commands as much as      %%
%% required.                                %%
%%                                          %%
%%%%%%%%%%%%%%%%%%%%%%%%%%%%%%%%%%%%%%%%%%%%%%

\address[id=aff1]{%                           % unique id
  \orgname{Faculty of Natural Sciences, Brandenburg University of Technology 
Cottbus--Senftenberg}, % university, etc
  \street{Gro\ss{}enhainer 57},                     %
  \postcode{01968}                                % post or zip code
  \city{Senftenberg},                              % city
  \cny{Germany}                                    % country
}
\address[id=aff2]{%
  \orgname{Department of Genomics, Faculty of Biotechnology, University of 
Wroc\l{}aw},
  \street{ul. Fryderyka Joliot-Curie 14a},
  \postcode{50-383}
  \city{Wroc\l{}aw},
  \cny{Poland}
}


%%%%%%%%%%%%%%%%%%%%%%%%%%%%%%%%%%%%%%%%%%%%%%
%%                                          %%
%% Enter short notes here                   %%
%%                                          %%
%% Short notes will be after addresses      %%
%% on first page.                           %%
%%                                          %%
%%%%%%%%%%%%%%%%%%%%%%%%%%%%%%%%%%%%%%%%%%%%%%

\begin{artnotes}
%\note{Sample of title note}     % note to the article
%\note[id=n1]{Equal contributor} % note, connected to author
\end{artnotes}

%\end{fmbox}% comment this for two column layout

%%%%%%%%%%%%%%%%%%%%%%%%%%%%%%%%%%%%%%%%%%%%%%
%%                                          %%
%% The Abstract begins here                 %%
%%                                          %%
%% Please refer to the Instructions for     %%
%% authors on http://www.biomedcentral.com  %%
%% and include the section headings         %%
%% accordingly for your article type.       %%
%%                                          %%
%%%%%%%%%%%%%%%%%%%%%%%%%%%%%%%%%%%%%%%%%%%%%%

\begin{abstractbox}

\begin{abstract} % abstract
\parttitle{Background} %if any
The quantitative real-time polymerase chain reaction (qPCR) and isothermal 
amplification are standard methods for quantification of nucleic acids. Numerous 
real-time read-out technologies with different technical foundation have been 
developed. However, the amplification curve analysis consists of cascaded steps, 
which are carried out similarly in all technologies. Despite the continuous 
interest in amplification based techniques, there are only few transparent tools 
for amplification data pre-processing. It is a major setback especially during 
development of new instruments, when the precise control on raw data is 
indispensable.

\parttitle{Results} %if any
$\emph{chipPCR}$ is an \textbf{R} package for pre-processing and quality 
analysis 
of amplification curve data from conventional quantitative polymerase chain 
reactions (qPCR) and quantitative isothermal amplification (qIA). The package 
takes advantage of \textbf{R}’s \emph{S4} object model. Algorithms to normalize 
amplification curves, to detect the start and end of an amplification reaction, 
to distinguish positive and negative amplification reactions and a powerful 
wrapper for smoothers are part of the software. Moreover, $\emph{chipPCR}$ 
employs 
a 5-point stencil for derivative interpolation, which has not been described 
for \textbf{R} to the best of our knowledge. The package contains several data 
sets, which were generated by helicase dependent amplification (HDA) or 
polymerase chain reaction (PCR) under various temperature conditions and 
detection systems, such as hydrolysis probes and intercalating dyes.

\parttitle{Conclusion}
We have developed $\emph{chipPCR}$, which is a versatile software tailored for 
the 
pre-processing of amplification curve data. Its utility is elaborated on both 
real and simulated data sets. The structure of the packages is open for 
integration to Web based and standalone \emph{shiny} applications. The 
\textbf{R} package along codes used for creation of figures used in publication 
is freely available.
\end{abstract}

%%%%%%%%%%%%%%%%%%%%%%%%%%%%%%%%%%%%%%%%%%%%%%
%%                                          %%
%% The keywords begin here                  %%
%%                                          %%
%% Put each keyword in separate \kwd{}.     %%
%%                                          %%
%%%%%%%%%%%%%%%%%%%%%%%%%%%%%%%%%%%%%%%%%%%%%%

\begin{keyword}
\kwd{R package}
\kwd{Amplification curve}
\kwd{Data analysis}
\kwd{Quantitative polymerase chain reaction}
\kwd{Quantitative isothermal amplification}
\end{keyword}

% MSC classifications codes, if any
%\begin{keyword}[class=AMS]
%\kwd[Primary ]{}
%\kwd{}
%\kwd[; secondary ]{}
%\end{keyword}

\end{abstractbox}
%
\end{fmbox}% uncomment this for twcolumn layout

\end{frontmatter}

%%%%%%%%%%%%%%%%%%%%%%%%%%%%%%%%%%%%%%%%%%%%%%
%%                                          %%
%% The Main Body begins here                %%
%%                                          %%
%% Please refer to the instructions for     %%
%% authors on:                              %%
%% http://www.biomedcentral.com/info/authors%%
%% and include the section headings         %%
%% accordingly for your article type.       %%
%%                                          %%
%% See the Results and Discussion section   %%
%% for details on how to create sub-sections%%
%%                                          %%
%% use \cite{...} to cite references        %%
%%  \cite{koon} and                         %%
%%  \cite{oreg,khar,zvai,xjon,schn,pond}    %%
%%  \nocite{smith,marg,hunn,advi,koha,mouse}%%
%%                                          %%
%%%%%%%%%%%%%%%%%%%%%%%%%%%%%%%%%%%%%%%%%%%%%%

%%%%%%%%%%%%%%%%%%%%%%%%% start of article main body
% <put your article body there>

%%%%%%%%%%%%%%%%
%% Background %%
%%
\section*{Background}

Quantitative polymerase chain reaction (qPCR) and quantitative isothermal 
amplification (qIA) are standard methods to amplify nucleic acids (e.g., genomic 
DNA, copy DNA). Main differences between qPCR and qIA are the thermal cycling 
conditions, the measuring point and the curve shape isothermal amplifications 
are monocyclic reactions at a constant temperature. Conversely, PCR is a 
polycyclic reaction with thermal cycling condition steps (denaturation, 
annealing, elongation) and measurements at discrete cycle steps. The curve shape 
of an isothermal amplification does not necessarily follow a S-shaped structure. 
The measurement of isothermal amplification reaction is time-based (continuous, 
not mandatory equidistant) in contrast to a cycle-based (discontinuous) 
measurement of qPCRs. The number of measure points in most cases higher than 
qPCR. Both amplification methods are used in different real-time monitoring 
technologies, such as our previously reported VideoScan technology, microfluidic 
systems, point-of-care devices and microbead-chip technologies and commercial 
real-time thermo cyclers \cite{chang_2012, roediger_highly_2013, 
roediger_bead_qPCR_2013, rodiger_nucleic_2014}. Real-time technologies enable 
the quantification of nucleic acids by calculation of specific curve parameters 
like the quantification point (Cq) and the amplification efficiency (AE) as 
described elsewhere \cite{ruijter_2013,tellinghuisen_2014}. Fundamental steps of 
amplification curve analysis are: (1) raw data read-in, (2) amplification curve 
pre-processing (e.g., noise reduction, outlier removal), (3) amplification curve 
processing (e.g., Cq and AE calculation), (4) post-processing and quantification 
of secondary parameters (e.g., Delta-Delta-Ct for gene expression analysis), 
and (5) data export and visualization.

  \textbf{R} is one of the most used tools in bioinformatics and is known to be 
an early adopter of emerging technologies \cite{pabinger_2014}. Most \textbf{R} 
packages focus on the read-in and (post)-processing of data sets, which 
originate from commercial qPCR systems. Sophisticated 
\textbf{R} packages for the steps 1. and 3.--5. are available 
from Bioconductor and CRAN. This includes packages to read-in qPCR data and to 
perform a selection of optimal reference genes for the data normalization. One 
of the most comprehensive packages is \emph{qpcR} \cite{ritz_2008}, which 
provides methods quantification and automatic optimization procedures to fit 
curves. Other packages were designed as frameworks for high-throughput 
analysis. The \emph{HTqPCR} package \cite{Dvinge_2009} can be used for the 
high-throughput data analysis (e.g., normalization, tested for differential 
expression) and visualization of different cards (samples). \emph{ReadqPCR} and 
\emph{NormqPCR} packages \cite{perkins_2012}) allow high-throughput qPCR data 
analysis. Main features include methods to select reference genes, 
normalization methods, and algorithms for differential expression between 
samples. The popular Delta-Delta-Ct ($ddCt$) algorithm \cite{livak_2001} is an 
approximation method to determine relative gene expression. qPCR is the method 
of choice for \emph{exact} gene expression analysis and an important step to 
validate experiments from microarrays, bead arrays and related technologies. 
Implementations are available from the \emph{ddCt}, \emph{NormqPCR} and 
\emph{qPCR.CT} package. The packages can be used to collect, analyze and 
visualize qPCRs. The \emph{SLqPCR} package can be used to normalize qPCR data 
and to compute the gene expression stability data. Other methods than the 
$ddCt$~algorithm have been proposed elsewhere \cite{heckmann_2011}.

However, novel developed qPCR and qIA technologies initially depend on 
independent tools to pre-process the raw data. Ideally, the experimenter 
receives a data output in an 
easily interpretable form. Pre-processing specifically addresses raw data 
inspection, steps to transform raw data in a compatible format for successive 
analysis steps (e.g., smoothing, removal of missing values), data reduction 
(e.g., removal of invalid sets) and data quality management. The data quality of 
experimental instruments is often not ready for end-user analysis and 
presentation. However, during the early stages of the development it is 
important to use as many raw data as possible. Pre-processing algorithms remove 
stochastic errors and artifacts (e.g., noise, photo-bleaching effects, degassing 
effects, different signal levels) illustrated in Figure~\ref{figure:problems}. 
Misinterpretations are more likely if non or manual (``arbitrary'') corrections 
are performed. A manual alteration is in contradiction to reproducible research. 
The presence of noise may lead to false conclusions and performance estimation. 
Noise is challenging because derivative processes as used for ``cycle of 
quantification'' methods (e.g., Second Derivative Maximum method) lead to an 
amplification of noise \cite{roediger_RJ_2013, larionov_2005, tuomi_2010, 
ruijter_2013, tellinghuisen_2014}.

The $\emph{chipPCR}$ package ($\emph{chipPCR}$: ``Lab-on-a-\textbf{Chip}'' \& 
PCR) was developed to automatize pre-processing, ease data analysis and to offer 
a quality control for the statistical data analysis of qPCR and qIA 
experiments. $\emph{chipPCR}$ is primarily targeted at users of experimental 
nucleic acid amplification systems and users who have access to the raw data of 
commercial systems. The following sections highlights pinnacles of the 
$\emph{chipPCR}$ package. Due to the vast amount of information we refer the 
interested reader to the manual for further information and use case scenarios.

\section*{Implementation}

$\emph{chipPCR}$ package was implemented in the \textbf{R} software environment 
for statistical computing and graphics. Supplemental materials, avaible as the
packages vignettes use Donald Knuth's literate 
programming principle~\cite{Knuth1984} to present the source code in the most convenient 
way for the users. The package is available at CRAN 
(\url{http://cran.r-project.org/web/packages/chipPCR/index.html}) along with an 
extensive documentation and examples. The latest source code is hosted at GitHub 
(\url{https://github.com/michbur/chipPCR}) with an issue tracker for bug reports 
or feature requests. For the present paper $\emph{chipPCR}$ version~0.0.8 was 
used. The naming convention in $\emph{chipPCR}$ is \textit{period.separated} 
names \cite{Baaaath_2012}. However, due to the organic growth of the package  
several elements violate this convention (e.g., functions names). 
$\emph{chipPCR}$ is a relative of \emph{MBmca} \cite{roediger_RJ_2013} and 
\emph{dpcR} but is mainly focused on the processing of amplification curve data.

  $\emph{chipPCR}$ uses \textbf{R}’s object model, designated \emph{S4} class 
system, which is implemented in the \emph{methods} package. In contrast to 
\textbf{R}’s \emph{S3} class system it is a requirement in \emph{S4} to declare 
classes, slots, methods relationships explicitly and to establish formal 
declarations of methods. This means that the number and types of slots in an 
instance of a class have to be established at the time the class definition and 
the objects from the class are validated against this definition and have to 
comply to it at any time. \emph{S4} methods are declared by calls to 
\textsl{setMethod} together with the names and signatures of the arguments. 
Signatures are used for identification of classes of one or more arguments of 
the methods. \textsl{setGeneric} can be used to declare generic functions. One 
disadvantage of \emph{S4} classes is a higher ``package author effort'' in 
comparison to \emph{S3} classes. However, \emph{S4} provides better assurance 
that each object in a class has the required slots, that data in the slots have 
consistent names and classes, and the possibility to include additional 
information (e.g., results, parameters) \cite{Karatzoglou_2004}.

Central functions of this package encompass: 

\begin{itemize}
\item \textsl{AmpSim}: a 5-parameter model for a 
S-shaped amplification curves accompanied by \textsl{AmpSim.gui} a \emph{shiny} 
GUI, for \textsl{AmpSim},
\item \textsl{bg.max}: a function to detect the start and end of an 
amplification reaction,
\item \textsl{CPP}: wrapper for several pre-processing functions,
\item \textsl{fixNA}: to impute missing values in a data column,
\item \textsl{inder}: for interpolating first and second derivatives 
interpolation using the five-point stencil (accompanied by 
\textsl{rounder} function),
\item \textsl{MFIaggr}: to analyze a bulk of replicates of an amplification 
reaction, 
\item \textsl{smoother}: to smooth the curve data by different methods 
(e.g., moving average, Savitzky-Golay smoothing).
\end{itemize}

In addition further (auxiliary)functions (\textsl{amptester}, 
\textsl{effcalc}, \textsl{humanrater}, \textsl{lm.coefs}, \textsl{normalizer}, 
\textsl{plotCurves}, \textsl{th.cyc}) typically used for post-processing (e.g., 
Cq calculation, amplification efficiency calculation) are available. For more 
information please refer to the the comprehensive supplement.

During the past years, several high-throughput technologies emerged, which 
require higher computing power. The core structures of $\emph{chipPCR}$ avoid 
loops to keep the code fast. Structures for parallel computing have been 
implemented in \textsl{smoother}. $\emph{chipPCR}$ includes a set of classes 
for plotting. The \textbf{R} graphics system allows fine control over the small 
details of plots \cite{Murrell_2012}. Although, we keep the output 
of our custom made plots simple, many parameters can be adjusted directly or by 
the ellipse parameter.

 $\emph{chipPCR}$ has no build-in functionality for data import or specialized 
parsers since \textbf{R} has numerous tools and axillary packages for customized 
parsers. Rather we have chose to rely entirely on \textbf{R} workspaces as 
default data format and standard import and export as described elsewhere 
\cite{RDCT2010c}. The results of $\emph{chipPCR}$ functions are usually S4 objects 
accompanied with specifically tailored summary and plot methods. To spread 
the use among novice and less routinized users we believe that a graphical user 
interface (GUI) is important. Currently there are several \textbf{R} GUI 
projects \cite{Valero_2012} to chose from. Some functionality of 
$\emph{chipPCR}$ (e.g., removal of missing values, smoothing, ...) originates 
from experimental \textbf{RKWard} plugins \cite{roediger_bead_qPCR_2013, 
roediger_RKWard_2013}. \textbf{RKWard} \cite{rodiger_rkward_2012} is a 
cross-platform GUI and integrated development environment for \textbf{R}. 
However, the functionality of $\emph{chipPCR}$ went beyond the functions of the 
plugins. Recently, \emph{shiny} emerged as a very interesting framework to build 
GUIs for the desktop (web browser) or services for interactive web applications, 
which can be deployed virtually anywhere. \emph{shiny} is a technology to build 
web applications with highly customizable widgets (e.g., sliders, plots, tables, 
reports) in a plugin-like approach. The combination \textbf{R} and \emph{shiny} 
technology allows a simple and repaid extension of the software. The 
\emph{shiny} applications update live and interactively. The user interfaces can 
be built entirely using R and operates in any R environment (see \emph{shiny} 
manual for details). Currently, \emph{shiny} GUIs for the functions 
\textsl{AmpSim}, \textsl{bg.max} (\textsl{AmpSim.gui} app) and 
\textsl{amptester} (\textsl{amptester.gui} app) are available (e.g., 
Figure~\ref{figure:browser}). Similarly to \textbf{QPCR} \cite{pabinger_2009} is 
\emph{shiny} in combination with \textbf{R} the foundation to build monolithic 
systems to parse, pre-process and analyze amplification curve data in a combined 
workflow. The GUIs are available on-line without requiring prior installation as 
seen in the code snippet below.

Data sets are an essential element of \textbf{R} packages \cite{hofmann_2013} 
and an indispensable requirement for reproducible research \cite{Leeper_2014}. 
The $\emph{chipPCR}$ package contains 22 data sets along with a description of 
the experimental procedure. The data sets stem from commercial qPCR cyclers 
(e.g., Bio-Rad iQ5, Bio-Rad CFX 96) and experimental systems. The later 
encompass the VideoScan heating/cooling unit (VideoScan HCU) 
\cite{roediger_highly_2013} and the capillary convective PCR (ccPCR) (see 
manual 
of $\emph{chipPCR}$). The experiments were performed under different condition 
including conventional qPCR, isothermal amplification by helicase dependent 
amplification (HDA) \cite{rodiger_nucleic_2014}. The overview can be 
conveniently invoked via \textsl{?chipPCR.datasets}

\begin{figure*}
\begin{verbatim}
# Load the shiny package (chipPCR should already be loaded).
# Run from a R console following commands.
require(shiny)

# Invoke the shiny AmpSim app in the default browser.
runApp(paste(find.package("chipPCR")[1],"/AmpSim.gui", sep = ""))

# Call shiny app AmpSim directly from gist
runGist('https://gist.github.com/michbur/e1def41598f1d0c1e2e6')
\end{verbatim}
\end{figure*}

\section*{Result and discussion}

$\emph{chipPCR}$ is an \textbf{R} package, which offers routines for 
pre-processing and quality management of experiments curves from real-time 
isothermal amplification and qPCR experiments. Implemented as \textbf{R} 
console tool it can be accessed and operated by an integrated development 
environment, RScript or related approaches. However, due to the use of the 
\emph{shiny} technology it can be accessed from a web browser from network 
connected computers or as local service. This allows to analyze, pre-process 
and to visualize results.

\subsection*{Multiple comparison of the cycle dependent variance of the 
amplification curves}

 \textsl{MFIaggr} is a powerful analytical and graphical tool for fast multiple 
comparison of the cycle dependent signal dispersion and distribution. The 
continuous response variable $y'$ is used to describe the relationships to one 
or more continuous predictor variables $y_1, ..., y_n$. Use cases include the 
comparison of independent reaction vessels or the analysis of replicate 
experiments. The basic idea is to analyze only a region of interest (ROI) from a 
data set. \textsl{MFIaggr} is a relative of the \textsl{MFIerror} function from 
the \emph{MBmca} package. However, this functions enables a fine grained 
analysis of specific parts of the curve data. The function returns an object of 
the class list with the columns ``Cycle'', ``Location'' (Mean, Median), 
``Deviation'' (Standard Deviation, Median Absolute Deviation) and ``Coefficient 
of Variation''. Using the option $rob = TRUE$ the median and the median absolute 
deviation (MAD) are calculated instead of the mean and standard deviation. 
\textsl{MFIaggr} has parameter $llul$ to define the lower and upper data limit 
(cycle) for a ROI. The results for the ROI can be invoke by $@stats$. The output 
includes the mean, median, standard deviation (sd), median absolute deviation 
(mad), inter quartile range (IQR), medcouple (robust measure of skewness), 
skewness (Pearson's second skewness coefficient; 
$skewness~=~3~(mean(x)~-~median(x))~/~sd(x)$), signal-to-noise ratio (SNR), 
variance-to-mean ratio (VRM), number of missing values (NAs) and results from a 
linear fit of the ROI (intercept, slope, r.squared). Moreover, we included 
the Breusch-Pagan test to test for heteroscedasticity in a linear regression 
model. In our example we analyzed the raw fluorescence from 96 replicates of a 
qPCR experiment for the human gene \textit{Vimentin}. The \textsl{MFIaggr} plot 
shows that the first ten cycles (noise) follow a normal distribution 
(Figure~\ref{figure:MFIaggr}). In contrast, the analysis of all cycles shows 
expectedly a distribution, which significantly differs from a normal 
distribution (Figure~\ref{figure:MFIaggr2}). Setting the option $CV = FLASE$ 
shows the relative standard deviation (RSD,~\%).

\begin{figure*}
\begin{verbatim}
par(las = 0, bty = "n", cex.axis = 1.2, cex.lab = 1.2, 
    font = 2, cex.main = 1.2, oma = c(1,1,1,1))

plot(MFIaggr(VIMCFX96_60[, 1], VIMCFX96_60[, 2:ncol(VIMCFX96_60)], 
     llul = c(1,10)), CV = FALSE)

plot(MFIaggr(VIMCFX96_60[, 1], VIMCFX96_60[, 2:ncol(VIMCFX96_60)], 
     llul = c(1,40)), CV = FALSE)
\end{verbatim}
\end{figure*}

Ideally the variance between the amplification curves is low. Other results of 
\textsl{MFIaggr} include the density analysis ($@density$), the quantile 
($@qqnorm.data$), and the results of the linear regression ($@lm.roi$) from the 
ROI. In particular, this function might be useful for quality management during 
the development of high-throughput technologies. An analysis via a the 
\emph{shiny} \textsl{MFIaggr.gui} app is shown in Figure~\ref{figure:MFIaggr2}.

\subsection*{humanrater - a human machine interface}

The \emph{chipPCR} package has numerous function to 
build streamlined work-flows for unsupervised pre-processing (see subsequent 
sections). However, as any unsupervised system is prone to errors we designed 
\textsl{humanrater}, which is a graphical interface to rate (``qualitative'' 
and/or ``quantitative'') curves interactively in an \textbf{R} GUI. All curves 
can be analyzed in random sequence, to reduce the risk of bias, or natural 
sequence. The number of replicate 
rating runs is user definable. A curve and prompts an input field for the user. 
This function can be used to 	 the human rating and the rating of a machine. 
Results of the analysis can be used for further processing such as Kappa 
statistics (Figure~S\ref{figure:MFIaggr2}).

\subsection*{Imputation of missing values in amplification curves}

  Experimental technologies may produce missing values (NA) at 
random due to sensor drop-outs or other technical difficulties. Many analytical 
functions stop to progress or discard entire data sets. This 
behavior is rational for unknown data structures. However, in case of 
amplification curve data it is justified to impute NAs because the structure 
generally resembles an S-shaped curve. Standard approaches include substitution 
with most frequent values, mean value imputation, last value carried forward, 
bootstrapping, or substitution by correlation with replicate measurements 
\cite{Harrell_2001}. In case of amplification curves other approaches are 
favorable. Particularly, the transitions phases (e.g., background phase to 
exponential phase) is prone to bias.

  The function \textsl{fixNA} package imputes missing values in a single column 
of data. The imputation is based either on a linear approximation or an 
approximation by cubic splines (default) (Figure~\ref{figure:fixNA}). 
Experiments showed that cubic spline interpolation yielded the most probable 
values and therefore led to the least effect on tested statistical parameters 
(Cq, background signal, Pearson correlation coefficient) the exponential phase 
and is therefore the recommended approach to remove missing values 
(Figure~\ref{figure:??}). We observed no significant bias by cubic spline 
interpolation (Table~S\ref{table:??}). The performance of \textsl{fixNA} using 
cubic splines was better than a linear interpolation (Figure~S\ref{figure:??}). 
However, the linear 
approximation might be applicable in measurements with high sample rates (e.g., 
isothermal amplification) (not shown). Any method requires a minimum number of 
data points as foundation for a meaningful imputation. \textsl{fixNA} tries to 
take care of such pitfalls. By rule of thumbs we determined that the number of 
missing elements in relation to the total number of elements. In case more than 
30~\% of all values are NAs gives \textsl{fixNA} a warning.

\subsection*{Smoothing and filtering}

  Amplification curve data of experimental thermo-cyclers may deliver results, 
which are hard to interpret due to noise. For data presentation it 
is often useful to smooth or filter the data. Smoothing and filtering are 
different approaches with a similar outcome. Both pre-process an input signal as 
output for subsequent analysis steps. Filtering uses methods of signal 
processing and takes a data input and apply a function to form an output. 
Smoothing in contrast uses statistical approaches, like local 
regression models (e.g., least squares estimate) or cubic splines. 
Therefore we developed the \textsl{smoother} function, which is a wrapper for 
smoother functions and filters commonly used to process amplification curve 
data. \textsl{smoother} inherited traits (Table~\ref{table:tab1})  of the 
parent functions. However, the functionality of \textsl{smoother} greatly 
outgrowths applications only in amplification curve analysis. Incorporating most 
of the best proven algorithms, we offer the user a powerful tool to access the 
methods while minimizing the drawback of learning syntax of specific functions. 
\textsl{smoother} was enhanced by functionality of \textsl{fixNA} and 
\textsl{CPP}. Figure~\ref{figure:smoother} shows results of the 
\textsl{smoother} function an amplification curve data. Many functions (e.g., 
Savitsky-Golay filter) of $\emph{chipPCR}$ assume uniform (equally spaced) 
sampling. Therefore it is recommended to pre-process the data to have equally 
spaced values. The function \textsl{smoother} and \textsl{CPP} (inherited from 
\textsl{smoother}) give a warning in such cases. The \textsl{smoother} function 
enables users to tune behavior of the chosen smoothing algorithm by using nearly 
all parameters available in called subroutines and at the same time uniforms 
input and output. It should be noted that smoothing may alter the curve shape 
and thus lead to artificial results. Smoothed data are easier to interpretable 
but introduce an additional bias to the pre-processed data. Therefore the prime 
use of smoothers is processing data for visualization purposes. However, it is 
not recommended to smooth signals unsupervised prior to statistical procedures 
(e.g., least-squares curve fitting). All smoothing algorithms are ``lossy'' to a 
certain extent and may change the curve shape significantly. In particular, the 
residual evaluation of a fit may lead to false prediction, because noise after 
smoothing may be mistaken for signal. Signals after curve smoothing can be used 
to locate peaks but it should cautiously be used to measure peaks.

\subsection*{Normalization of amplification curve data}

It is a common characteristic of amplification curve data that the fluorescence 
values between samples vary due to high background, sample inhomogeneities and 
variances in dye quantities (Figure~\ref{figure:normalization}~\emph{A}). 
Therefore, normalization of amplification curve data is a common task during the 
data analysis. To scale the fluorescence between 0 and 1 a \emph{Min-Max 
normalization} (Equation~\ref{eq:normalization}) can be used 
\cite{roediger_RJ_2013}. We propose an alternative normalization based on 
quantiles (Equation~\ref{eq:quantile_normalization}). Quantiles are less 
affected by outliers. The method can be invoked by the parameter $norm = 
"luqn"$. Although this does not scale all values between zero and one we found 
it to be useful for noisy data. The parameter $qnL$ is symmetrically used to set 
the level for the quantiles. By default the 3~\% and 97~\% quantiles are used 
for the normalization. In addition a normalization to maximum 
(Equation~\ref{eq:max_normalization}, 
Figure~\ref{figure:normalization}~\emph{D}) and by standard score 
(Equation~\ref{eq:zscore_normalization}, 
Figure~\ref{figure:normalization}~\emph{F}).

\begin{equation} \label{eq:normalization}
RFU_{minmax} = \frac{RFU - \min(RFU)}{\max(RFU) - \min(RFU)}
\end{equation}

\begin{equation} \label{eq:max_normalization}
RFU_{max} = \frac{RFU}{\max(RFU)}
\end{equation}

\begin{equation} \label{eq:quantile_normalization}
RFU_{luqn} = \frac{RFU - Q_{p}(RFU)}{Q_{1 - p}(RFU) - 
Q_{p}(RFU)}
\end{equation}

\begin{equation} \label{eq:zscore_normalization}
RFU_{zscore} = \frac{RFU - \bar{x}_{RFU}}{s_{RFU}}
\end{equation}

  The slope in a curve can be corrected by a linear regression. \textsl{CPP} 
offers four linear regression models to calculate the slope based on the 
background range. This includes a ordinary least squares method (\textsl{lm}, 
\emph{stats}) but also three robust methods. The robust regression methods are 
considered to be less vulnerable to outliers. This feature is especially useful, 
when the background range contains considerable noise. The methods are (I) a 
nonparametric rank-based estimator \cite{Kloke_2012}, (II) quantile regression 
\cite{Koenker_2008} and (III) a MM-type estimators for linear regression 
\cite{Todorov_2009}. By default the MM-type estimator is used. In all cases 
takes \textsl{CPP} a defined range of the amplification curve to extrapolate the 
linear trend over the entire data set. However, this step his to be performed 
with caution since this operation effects the amplification efficiency. The 
background is assumed to be constant for the entire measurement. 

\subsection*{Interpolation of derivatives}
 
  Many methods for curves analysis require the calculation of derivatives. It 
is 
possible to solve this by fitting a curve to a function and performing symbolic 
derivation. Unfortunately, this approach causes information loss through the 
fit 
and unnecessary adds additional assumptions regarding the relation between 
cycle number and fluorescence level. Hence, we integrated the \textsl{inder} 
function. \textsl{inder} (``in'' and ``der'' = interpolate derivatives) finds 
numeric derivatives by a five-point stencil, a commonly used finite difference 
method. These methods approximate derivative in a given point by adding up 
products of nearby values of function and their weights~\cite{Dahlquist_2008}. 
This function can be used to estimate the approximate cycle of quantification 
(Cq). Differentiation is a method for background suppression and reduction of 
the inter sample background amplitude variations 
(Figure~\ref{figure:inder_fit}~A~and~B). Smoothing may enhance the calculation 
of derivatives calculation and optimize the signal-to-noise ratio. Therefore, 
we 
implemented spline interpolation. Friedman's SuperSmoother is also implemented. 
However, the use of this smoother is limited for the use in other functions 
such 
as \textsl{bg.max}. The parameter $Nip$ (default $Nip = 4$) is used to define 
how often an interpolation takes place at n equidistant points within the first 
and the last cycle. A high Nip may improve the precision. However, $Nip$ less 
than 2 and higher than 20 are not meaningful for conventional qPCR with 30 to 
50 
cycles. In context of qIA, a higher $Nip$ might be appropriate.

\subsection*{Estimating the start and the end of amplification process}

  The correction of background fluorescence is an important steps in 
amplification curve analysis. Background herein refers to a level of 
fluorescence measured before any specific amplification is detectable. It has 
influence on amplification efficiency calculation and model fitting 
\cite{tuomi_2010, rutledge_2008, ruijter_2009}. There are numerous ways of 
compensating background noise, ranging from simple to very sophisticated 
solutions. For a detailed description see the manual of the \textsl{chipPCR} 
package v.~0.0.8 in section \textsl{bg.max} ``Details''. The function 
\textsl{bg.max} is a fit-free method to estimate the range of the background 
starting from the raw data. The raw data (e.g., fluorescence intensity) 
measured 
after each step (cycle or time point) follow a non-linear progress. The 
\textsl{bg.max} algorithm assumes that the signal difference $\Delta{y}$ of 
successive cycles ($\Delta{y} = y_{n + 1} - y_n$) in the linear ground phase is 
approximately constant. After transition in the early exponential phase the 
signal changes drastically. First data are smoothed by Friedman’s ’super 
smoother’ (\textsl{supsmu}, \emph{stats}). Thereof the approximate first and 
second derivative are calculated. The difference of cycles at the maxima of the 
first and second approximate derivative and a correction factor are used to 
estimate the range before the exponential phase. This simple function finds the 
background range without modeling the function. The start of the background 
range is defined be a fixed value. Since many signals tend to overshot in the 
first cycles a default value of two is chosen. \textsl{bg.max} tries also to 
estimate the end of an amplification reaction (Figure~\ref{figure:bgmax}).

\subsection*{Proposed work-flow}

The function \textsl{CPP} (``Curve Pre-Processor'') is a wrapper for several 
functions described before in this study. \textsl{CPP} 
can be considered as proposed workflow for an amplification curve 
pre-processing. In particular, the functions \textsl{fixNA} $\rightarrow$ 
\textsl{bg.max} $\rightarrow$ \textsl{smoother} $\rightarrow$ 
\textsl{normalizer} $\rightarrow$ \textsl{amptester} were cascaded. A typical 
output of \textsl{CPP} is shown in Figure~\ref{figure:fixNA_CPP} and 
Figure~\ref{figure:normalization}.

\section*{Conclusions}

To the best of our knowledge is the $\emph{chipPCR}$ is the first \textbf{R} 
package for the pre-processing and raw data quality analysis of amplification 
curve data. Others and we share the philosophy that software in research should 
not be a black-box and but part of an transparent ecosystem for reproducible 
research and exploration of new information from existing data 
\cite{Thioulouse_2010, roediger_RJ_2013, hofmann_2013, Leeper_2014, liu_2014}. 
In particular, the smoothing methods and algorithms for the derivatives are 
often not documented and therefore hard to judge by others. The $\emph{chipPCR}$ 
package is open source (with GPL-3 license) and freely available through CRAN. 
$\emph{chipPCR}$ primarily targets pre-processing. However, the $\emph{chipPCR}$ 
has also implementations to process amplification curve data. These can be 
embedded in customized routines with other packages mentioned in this study 
(see supplement). In fact, the packages \emph{dpcR} and \emph{MBmca} depend on 
$\emph{chipPCR}$ technology. We showed that $\emph{chipPCR}$ is build from 
smaller blocks and show how users can do estimation of background by hand, 
solely by \textsl{inder}, \textsl{smoother} (\textsl{smoother} will be a method 
of smoothing in \textsl{inder}) and by putting data in \textsl{bg} object with 
\textsl{summary-der} for $SDm$ and $SDM$. It should be quite easy even for an 
inexperienced user. We can claim that modular structure of $\emph{chipPCR}$ 
package allows user to perform flexible data analysis adjusted to their needs. A 
limitation of all \textbf{R} packages related to qPCR and qIA is the lack of a 
comprehensive GUI. We argue that that our \emph{shiny} GUI approach might be a 
promising approach. Similar applies to a standard for data exchange. A 
standardized might be embedded in structures for models (e.g., Predictive Model 
Markup Language (PMML) \cite{Guazzelli_2009}) and data exchange (e.g., XML-based 
Real-Time PCR Data Markup Language (RDML) \cite{lefever_2009}, binary formats 
\cite{michna_2013}). However, work on this hasn't started yet. In 
high-throughput systems might computing speed be an issue. Therefore further 
parallelization of computing processes as described elsewhere 
\cite{Schmidberger_2009, boehringer_2013} is part of future developments. 

\section*{Availability and requirements}
Project name: chipPCR, 
Project homepage (development):
\url{https://github.com/michbur/chipPCR}, 
Project homepage at CRAN: 
\url{http://cran.r-project.org/web/packages/chipPCR/index.html}, 
Operating System: Platform independent, 
Other requirement: R 3.0.0 or higher, 
License: GPL-3

%%%%%%%%%%%%%%%%%%%%%%%%%%%%%%%%%%%%%%%%%%%%%%
%%                                          %%
%% Backmatter begins here                   %%
%%                                          %%
%%%%%%%%%%%%%%%%%%%%%%%%%%%%%%%%%%%%%%%%%%%%%%

\begin{backmatter}

\section*{Competing interests}
The authors declare that they have no competing interests.

\section*{Authors' contributions}
SR conceived of the study, and participated in its design and coordination and 
wrote the manuscript. SR and MB jointly developed the software. All authors read 
and approved the final manuscript.

\section*{Acknowledgment}
Part of this work was funded by the BMBF InnoProfile-Projekt 03 IPT 611X. 
Grateful thanks belong to all authors of the cited \textbf{R} packages, the 
\textbf{R} community and \textbf{RKWard} developers. We would like to thank 
Claudia Deutschmann (Brandenburg University of Technology Cottbus - Senftenberg, 
Germany), Ralf Himmelreich (Fraunhofer ICT-IMM, Germany) and Katharina Klat 
(Darmstadt University of Applied Sciences, Germany) for the contribution of 
data.
%%%%%%%%%%%%%%%%%%%%%%%%%%%%%%%%%%%%%%%%%%%%%%%%%%%%%%%%%%%%%
%%                  The Bibliography                       %%
%%                                                         %%
%%  Bmc_mathpys.bst  will be used to                       %%
%%  create a .BBL file for submission.                     %%
%%  After submission of the .TEX file,                     %%
%%  you will be prompted to submit your .BBL file.         %%
%%                                                         %%
%%                                                         %%
%%  Note that the displayed Bibliography will not          %%
%%  necessarily be rendered by Latex exactly as specified  %%
%%  in the online Instructions for Authors.                %%
%%                                                         %%
%%%%%%%%%%%%%%%%%%%%%%%%%%%%%%%%%%%%%%%%%%%%%%%%%%%%%%%%%%%%%

% if your bibliography is in bibtex format, use those commands:
\bibliographystyle{bmc-mathphys} % Style BST file
\bibliography{roediger-burdukiewicz}      % Bibliography file (usually '*.bib' )

% or include bibliography directly:
% \begin{thebibliography}
% \bibitem{b1}
% \end{thebibliography}

%%%%%%%%%%%%%%%%%%%%%%%%%%%%%%%%%%%
%%                               %%
%% Figures                       %%
%%                               %%
%% NB: this is for captions and  %%
%% Titles. All graphics must be  %%
%% submitted separately and NOT  %%
%% included in the Tex document  %%
%%                               %%
%%%%%%%%%%%%%%%%%%%%%%%%%%%%%%%%%%%

%%
%% Do not use \listoffigures as most will included as separate files

\section*{Figures}
  \begin{figure}[h]
  \caption{\label{figure:problems}
  \csentence{Analysis and interpretation of real-time amplification curves.} 
Analysis and interpretation of real-time amplification curves. \emph{(A)} The 
fluorescence values are plotted against the cycle. The amplification curve has a 
sigmoidal shape. Amplification curve raw data are affected by many influences. 
This includes noise introduced by the detection system and sensor errors. 
Measurements can occasionally contain missing values (``NA'') and outliers 
(orange circle). Outliers are often present in the first cycle due to sensor 
adjustments. The signal difference between the background phase (first cycles) 
and the plateau phase (last cycles) can be expressed as signal-to-noise ratio 
(SNR). The SNR between different between samples (e.g., black line and red line) 
can vary. For interpretation it is better to compensate the 
differences. Negative samples (blue line) need to be (automatically) identified. 
\emph{(B)} pre-processed raw data. ``NAs'' were imputed and the noise slightly 
removed. The curves were adjusted to have the same baseline and plateau level. 
The quantification point (Cq) of the positive reactions are determined in the 
exponential phase (``Threshold method'' is used in this example). Negative 
sample are automatically set to zero.
}
      \end{figure}

\begin{figure}[h]
  \caption{\label{figure:browser}
  \csentence{Locally running \emph{shiny} \textsl{AmpSim.gui} app.} \emph{(Top)} 
The plot of the \textsl{AmpSim.gui} is shown in a standard browser 
(\textbf{Iceweasel}, v. 29.0.1) along with the parameters (left panel). The code 
(``server.R'', ``ui.R'') of the \emph{shiny} app is accessible too. All 
parameters (e.g., Cq value, Baseline) of the \textsl{AmpSim} function are 
accessible. \emph{(Bottom)} In addition shows \textsl{AmpSim.gui} the plot 
output and the textual results of \textsl{bg.max}}
      \end{figure}
      
\begin{figure}[h]
  \caption{\label{figure:MFIaggr}
  \csentence{Amplification curve analysis of the \textsl{VIMCFX96\_60} data 
set.} The cycles 1 -- 10 were selected as region of interest (ROI). The function 
\textsl{MFIaggr} was used to analyze the variance of a 96-well plate cycler 
(Bio-Rad CFX96) and EvaGreen\textregistered for detection (left panel). The mean 
and the median 
were almost identical. The density plot (right upper panel) and 
quantile-quantile analysis (right lower panel). Both analytical plots indicate 
that the data of the background range are normal distributed. Note that both the 
$median \pm mad$ and the $mean \pm sd$ have similar results. }
      \end{figure}       

\begin{figure}[h]
  \caption{\label{figure:smoother}
  \csentence{Smoother and filter methods of the $\emph{chipPCR}$ package.} 
\emph{(A)} Raw data were generated using the \textsl{AmpSim} simulation 
function. \emph{(B)} The difference of the raw data to the smoothed data was 
plotted. ``savgol'' (Savitzky-Golay Smoothing), ``lowess'' (locally-weighted 
polynomial regression), ``mova3'' (moving average with window size of 3), 
``smooth'' (cubic smoothing spline), ``spline'' (Interpolating cubic spline), 
``supsmu'' (Friedman's SuperSmoother), ``whit1'' (weighted Whittaker smoothing 
with a finite difference penalty of order 1), ``whit2'' (weighted Whittaker 
smoothing with a finite difference penalty of order 2). The ``savgol'', 
``smooth'', ``spline'' ``whit1'' , and ``whit2'' nearly preserved the original 
curve. The other functions resulted in alteration in the transition phases of 
the amplification curve. Optimized time series smoother, like the Kalman 
filter \cite{Tusell_2010}, are not yet integrated. }
      \end{figure}       
     
%%%%%%%%%%%%%%%%%%%%%%%%%%%%%%%%%%%
%%                               %%
%% Tables                        %%
%%                               %%
%%%%%%%%%%%%%%%%%%%%%%%%%%%%%%%%%%%

%% Use of \listoftables is discouraged.
% %%
\section*{Tables}
\begin{table*}[b]
\caption{\label{table:tab1} \csentence{Smoothing and filter methods of the 
$\emph{chipPCR}$ package}. The parameter $lowess$ for LOWESS smoother 
(locally-weighted polynomial regression) can be tuned by the parameters $f$ and 
$iter$. The parameter $mova$ for moving average can be tuned by the parameter 
$movaww$. $movaww$ is the window size used for the moving average. The parameter 
$savgol$ for Savitzky-Golay smoothing filter can be tuned by the parameter $p$ 
(see \textsl{sgolayfilt} (\emph{signal}) for details). The parameter $smooth$ 
for cubic spline smooth can be tuned by the parameter $df.fact$. A $df.fact$ 
value of 1 will leave the raw data almost unaffected while a value 0.5 will 
smooth the curve considerably. The parameter $spline$ for standard cubic spline 
smooth has currently no additional parameter. The parameter $supsmu$ for 
Friedman's SuperSmoother can be tuned by the parameter \textsl{span}. The 
parameter $whit1$ (first order finite difference penalty) and $whit2$ (second 
order finite difference penalty) for Weighted Whittaker smoother smoothing 
filter, derived from the \emph{ptw} package, can be tuned by the parameter 
$lambda$. For further details on the smoothers refer to the documentation of the 
parent functions.
}
\begin{tabular}{lccc}
  \hline
  Method & Parameter & value & Parent\\ \hline
  LOWESS & $lowess$ & \textit{f} & \textsl{lowess}, \emph{stats} \\
  Cubic spline & $smooth$ & \textit{df.fact} & \textsl{smooth.spline}, 
\emph{stats} \\
  Interpolating Splines & $spline$ & - & \textsl{spline}, \emph{stats} \\
  Friedman's ``super smoother'' & $supsmu$ & \textit{span} & \textsl{supsmu}, 
\emph{stats}\\
  Savitsky-Golay & $savgol$ & - & \textsl{sgolayfilt}, \emph{signal} \\
  Moving Average & $mova$ & \textit{movaww} (3, 5, ...) & \textsl{filter}, 
\emph{stats} \\
  Whittaker & $whit1$, $whit2$ & \textit{lambda} & \textsl{whit1}, 
\textsl{whit2},
\emph{ptw}  \\
  All smoother & $all$ & defaults &
  \\ \hline
\end{tabular}
\end{table*}

%%%%%%%%%%%%%%%%%%%%%%%%%%%%%%%%%%%
%%                               %%
%% Additional Files              %%
%%                               %%
%%%%%%%%%%%%%%%%%%%%%%%%%%%%%%%%%%%

\section*{Additional Files}
  \subsection*{Additional file 1 --- Sample additional file title}
    Additional file descriptions text (including details of how to
    view the file, if it is in a non-standard format or the file extension).  This might
    refer to a multi-page table or a figure.



\end{backmatter}
\end{document}
